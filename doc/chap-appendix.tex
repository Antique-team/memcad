\appendix
\chapter{Appendix}

This appendix provides information about the syntactic transformation
performed on the AST prior to analysis, and on the techniques for
troubleshooting analyses.

\section{AST transformations}
The opam package \sfile{clangml-transforms} allows to transform step by
step the full blown \sfile{clang} C AST into the leaner and simpler AST
that \memcad uses internally.
This structure greatly simplifies the internal design of the abstract
interpreter as it alleviates the need to handle many C features at that
level.
Those transformations are listed and explained hereafter, in the order
in which they are applied to the \sfile{clang} AST, and prior to the
analysis.
\begin{enumerate}
\item \texttt{RemoveCast}: \\
  Removes implicit and explicit cast operations.
\item \texttt{MoveFunCall}: \\
  Expands the \sstmt{return} statements into assignments, by
  creating a temporary variable for each function call, and before
  the call sites.
\item \texttt{NoContinueInFor}: \\
  Filters out codes that use \sstmt{continue} statements (that are
  currently unsupported).
\item \texttt{RemoveCommaBinop}: \\
  Transforms the expressions with coma operators into sequences of
  statements.
\item \texttt{BreakOpAssign}: \\
  Decompose binary operators combined with assignment into expressions
  with classical binary operators.
\item \texttt{BreakMultiAssign}: \\
  Decomposes multiple assignments into sequences of assignments.
\item \texttt{SimplifyIf}: \\
  Decomposes complex \sstmt{if} statements into an imbrication of
  \sstmt{if} statements.
\item \texttt{ForToWhile}: \\
  Transforms \sstmt{for} loops into \sstmt{while} loops.
\item \texttt{DoWhileToWhile}: \\
  Transforms \sstmt{do while} loops into \sstmt{while} loops.
\item \texttt{SwitchToIf}: \\
  Transforms some \sstmt{switch} statements into sequences of \sstmt{if}
  statements
  (this transformation does not handle all \sstmt{switch} statements,
  and will crash in too complex cases).
\item \texttt{MoveAssign}: \\
  Moves assignment statements outside of conditions.
\item \texttt{ReplaceFunCall}: \\
  Completes the decomposition of function calls performed by
  transformation \texttt{MoveFunCall}.
  \commentout{Maybe those two transforms can be combined into a single one.}
\item \texttt{SimplifyDeclStmt}: \\
  simplifies multiple declaration statements.
\item \texttt{AssignCond}: \\
  Transforms assignment involving conditional expressions using
  conditional statements.
\item \texttt{LiftConditionals}: \\
  Transforms conditional operators into \sstmt{if} statements.
\item \texttt{PostIncrDecr}: \\
  Decomposes post incrementation (\verb#x++/x--#).
\item \texttt{PreIncrDecr}: \\
  Decomposes pre-incrementation (\verb#++x/--x#).
\item \texttt{SplitInitialisers}: \\
  Splits declaration with initializers into separate declarations
  and assignments.
\end{enumerate}


\section{Preparing some C source code for analysis}
In this section, we provide guidance how to carry out the analysis of
real-world C programs with \memcad, based on our experience in doing
so for several data structure libraries and programs manipulating
complex structures:
\begin{enumerate}
\item \textbf{Compile the target program or the library.} \\
  The \memcad analyzer uses the \sfile{clang} parsing stage, so it
  is very unlikely that a program that cannot be parsed by that
  compiler can be analyzed successfully.
  We advise using the options \soption{-W} and \soption{-Wall} to
  trigger most warnings.
  Code raising warnings may challenge the AST transformation stages.
\item \textbf{Set up a ``harness'' \sstmt{main} function (if necessary).} \\
  As the \memcad analyzer implements a whole program analysis, it requires
  an entry point as part of the analysis options.
  Therefore, when trying to verify specific properties of a program
  (\eg, that it satisfies some post-condition, when ran under a specific
  pre-condition), these need be specified using the \memcad specific
  commands.
  Thus, in the case of a set of library functions, we advise to set
  up a harness including a dummy main function, with pre-conditions
  and post-conditions for the functions to verify.
  In the case of a whole program analysis, no harness is necessary.
\item \textbf{Launch the analysis.}
  Basic options should include the file to analyze, the entry point,
  and an indication of how the inductive predicates should be fixed
  (either in an external file, or synthesized automatically).
\end{enumerate}
In case the analysis fails due to an error localized in the frontend
and transformation stages, the best solution is to rewrite unsupported
languages features, if possible.

When the analysis fails due to an error raised during the analysis
phase, the following actions may help troubleshooting the problem,
and either finding a set of options that will allow the analysis to
run properly, or make a more precise bug report:
\begin{itemize}
\item Analyzing the program with a stronger pre-condition, possibly
  narrowing it to the verification of a single execution trace.
  This technique alleviates the need for certain operations in the
  analysis, but still make use of others, thus it helps localizing
  where the failure stems from.
\item Executing the compiled program and its test suite (if available),
  and checking the correctness of the concrete outputs.
\item Executing the program under valgrind\footnote{
    \url{http://valgrind.org/}} to look for memory problems
  and correct them if possible.
\end{itemize}


\section{Running MemCAD in the debugger}
In this section, we provide instructions to run the \memcad
analyzer under the OCaml debugger (\sexec{ocamldebug}).

The two following options of the ``\sexec{analyze}'' executable
allow to exclude the initial parsing phase, and run the analysis
itself separately, and under a debugger:
\begin{itemize}
\item[\soption{-dump-parse}]: \\
  Dumps the AST into a file, after the \sexec{clang} parsing and the
  application of all the AST transformations;
\item[\soption{-load-dump}]: \\
  Loads an AST that was dumped by the \soption{-dump-parse} option.
\end{itemize}

To compile the bytecode-debug version of the analyzer (called
\sexec{danalyze}), run the following bash commands (the export
command is needed because of the apron library):
\begin{verbatim}
$ export LD_LIBRARY_PATH=`opam config var \
    share`/apron/lib
$ make danalyze
\end{verbatim}

Second, you need to dump out the AST of the C program to analyze.
To do this, execute the following bash command:
\begin{verbatim}
$ ./danalyze -header bench/bigrt/for_gdsl.h \
    -very-silent -nd-box -a -no-latex \
    -dump-parse to_analyze.c
\end{verbatim}
This command silently creates the file 'to\_analyze.c.dump'.

Finally, you need to run the bytecode-debug analyzer in ocamldebug
and use the previously dumped AST.
All these steps are necessary so that there are no more calls to
\sexec{clangml} during the analysis
(\sexec{ocamldebug} does not support it and would hang).

\begin{verbatim}
$ ocamldebug ./analyze.d.byte -header \
    bench/bigrt/for_gdsl.h -very-silent \
    -nd-box -a -no-latex -load-dump \
    source_code_to_analyze.c
\end{verbatim}

This command will silently read the file 'to\_analyze.c.dump',
instead of parsing the C file with clangml then applying all the
clangml-transforms.
You can then use any ocamldebug command\footnote{
\url{http://caml.inria.fr/pub/docs/manual-ocaml/debugger.html}}.
