%--------------------------------------------------------------------
% Brief general description of the structure of the analyzer
%--------------------------------------------------------------------
\chapter{Introduction to the MemCAD analyzer and basic usage}

\section{Informal presentation}
\label{s:2:1:informal}
\memcad stands for {\bf Mem}ory {\bf C}omposite {\bf A}bstract {\bf D}omain.
It is an experimental tool for program verification, that focuses mainly
on memory properties.

\memcad is a static analyzer for C programs, specialized in the inference
and verification of memory properties.
It performs a forward analysis by abstract interpretation.
More precisely, it starts from abstract pre-conditions and computes abstract
post-conditions.
Unless specified otherwise, the pre-condition assumes an empty memory state
(no variable is defined)
The analysis is designed to be sound, which means that the invariants it
computes should over-approximate the concrete behaviors of the analyzed
program
(though, note that \memcad is offered with no guarantee, even though the
design intent is to ensure its soundness ---\eg, the analyzer may not be
free of implementation bugs).
%% XR: TODO: one thing to take care of is the semantics of malloc
For instance, when a new local variable is defined and before it is
uninitialized, it is assumed to hold any possible value.
To allow for the verification of a code fragment under some assumption
(\eg, that a function is called in a state where some structure are
already built and stored in memory), the analyzer features specific
assumption commands.

During the analysis, \memcad attempts to prove the absence of memory
errors and assertion violations.
In particular, it attempts to prove:
\begin{itemize}
\item the absence of memory errors, such as the dereference of null or
  invalid pointers;
\item the absence of accesses (read or write) outside the bounds of
  memory blocks (that includes array out of bound errors);
\item that regular assertions are satisfied;
\item that structural assertions (that are described in a specific
  language) are also satisfied, which means that certain structural
  invariants hold.
\end{itemize}
The main use of \memcad consists in verifying the preservation of
structural assertions such as the preservation of well-formed dynamic
structures such as lists, trees, possibly nested into arrays.

The analysis is conservative and may fail to prove properties that hold
in all concrete executions.
Messages reporting the analyzer cannot prove the correctness of some
operations do not necessarily mean that there exists a concrete execution
where they do not hold, or even that the analyzer cannot prove them under
different settings.
In many cases, they stem for imprecision in the analysis, that could either
be solved using other settings, or require some deeper refinements of the
abstract interpreter implemenation (addition of an abstract domain,
refinement of abstract transfer functions and analysis algorithms).
In the current version of the \memcad implementation, shape properties
(well formedness of structures with respect to purely pointer properties)
and basic numerical properties (linear constraints) can be specified and
are handled rather precisely.
Also, set abstraction allow to verify the preservation of shared
data structures.
The analysis handles basic relations between pointer properties, numerical
properties and sharing properties, but cannot handle precisely all such
properties (the current version is not able to generalize numerical
invariants on red-black trees, for instance).

The abstract domain of \memcad is highly parametric and modular.
The analyzer relies on a library of abstract domains that can often
be enabled or disabled independently.
Some domains require some parameterization, which can be performed
either automatically or manually.

The \memcad analyzer inputs C programs, and optional user-supplied
description of data-structures of interest.
The purpose of the data-structure descriptions is to parameterize the
abstract domain of the analyzer.
The data-structure descriptions can be inferred automatically when the
structures involve no sharing or pointer relations (as in doubly-linked
lists or trees with parent pointers).
In the other hand, basic structures such as singly-linked trees or
binary trees can be inferred from their type definitions.
Not all C features are supported and the programs to analyze must be
compilable by the clang compiler.
Moreover, \memcad does not aim at handling all features of the C
programming language.
The following C features are not currently supported by \memcad,
although support of several of these could be added easily (the
main purpose of \memcad is to design, implement and
assess memory abstractions, not numerical abstractions):
\begin{itemize}
\item floating point variables;
\item inlined assembly code;
\item recursive functions;
\item function pointers;
\item pointers used as arrays (though core of the analyzer could
  handle this feature well, the frontend and typing phase currently
  do not).
\end{itemize}

In the following sections, we illustrate basic use of the \memcad
static analyzer.

\section{Use of \memcad to verify basic properties~/~programs}
\label{sec:usage-example}
We consider a very simple program manipulating singly-linked lists to
illustrate a first analysis example, and assume that this example
program is stored in a file named "\texttt{list.c}" located in the
current directory:
\begin{verbatim}
typedef struct elt {
  struct elt* next;
  int data;
} elt;

typedef elt* list;

void main() {
  list l;
  int i = 0;
  _memcad("add_inductive(l,elt)");
  while(l != 0){
    l->data = i++;
    l = l->next;
  }
  _memcad("check_inductive(l,elt)");
}
\end{verbatim}

First, let us note that the meta-command \verb#_memcad# is used to
control the analysis, and state assumptions and proof goals using a
meta-syntax specific to the analyzer.
Two instances of this command are used here:
\begin{itemize}
\item the first one consists of the meta-command \verb#add_inductive#
  and requires the analyzer to assume at the point where it appears the
  value stored in variable \verb#l# satisfies some inductive predicate in
  separation logic named \verb#elt# (this predicate describes a structure
  that can either be inferred from type definitions or set manually; more
  details on this predicate will be provided below);
\item the second one consists of the meta-command \verb#check_inductive#
  and requires the analysis to prove that the value stored in variable
  \verb#l# still satisfies the inductive predicate named \verb#elt#.
\end{itemize}
Essentially, those two commands are used here so as to specify an abstract
pre-condition (to assume) and an abstract post-condition (to prove).

Then, the analysis is launched by the command below:
\begin{verbatim}
./analyze -a -auto-ind list.c
\end{verbatim}
The \soption{-a} option instructs the executable to perform the analysis.
The \soption{-auto-ind} option activates automatic inference of the set
of inductive definitions to be used for the analysis from the type
definitions in file \sfile{list.c}.
This generates one inductive definition \verb#elt# to be used as a
parameter of the abstract domain before actually launching analysis.
Thus, the execution is split into two phases:
\begin{enumerate}
\item collection of inductive definitions computed from the type
  definitions (under the assumption that no sharing is used), and
  instantiation of the abstract domain;
  essentially, this phase computes an inductive definition in separation
  logic of the form below:
  \[
  \begin{array}{lccl}
    \icallpz{\alpha}{\indelt}
    & ::= & & (\emp, \alpha = 0) \\
    & & \logor &
    (\alpha \cdot \fldnext \relpt{} \beta_0
    \lsep \alpha \cdot \flddata \relpt{} \beta_1 \\
    & & & \lsep \icallpz{\beta_0}{\indelt}, \alpha \not= 0) \\
  \end{array}
  \]
  (internally, the analysis takes the memory layout into account
  and utilizes numeric offsets rather than field names)
\item forward abstract interpretation of the program using this
  inductive definition as a parameter of the abstract domain, and
  starting from the entry point of the program with an empty memory
  state (no variable, and no heap space);
  this phase over-approximates soundly each variable definition and
  program statement;
  while doing so, the \verb#_memcad# meta-commands respectively
  lets the analysis make a memory assumption (which would be satisfied
  if replaced by any code fragment that onstructs a valid structure,
  that can be described by \verb#elt#), and requires it to verify
  a memory property.
\end{enumerate}
Note that an automatically generated inductive definition is named
after the name of the struct it is derived from.
In our example, the struct is named \verb#elt#, so \verb#elt# must
be used as the inductive name in both the \verb#add_inductive# and
in the \verb#check_inductive# \memcad commands.

The analyzer output will end like this:
\begin{verbatim}
[...]
Final status report (0.1901 sec spent)
- assertions proved:     1 (context)
- assertions not proved: 0 (context)
\end{verbatim}
If the count on the line 'assertions not proved' is non zero, it means
the program could not be verified (\ie, at least one memory assertion
could not be proved).

The automatically generated inductive definition can be found in the file
named 'auto.ind'.
The syntax of inductive definition files is described in details
in section \sref{3:3:ind}.

\section{How to}
We now show a couple of additional examples of analyses, with very general
or very specific pre-conditions.

\subsection{Analyze a program in the abstract}
%% TODO: does it prove the balancing ?
The following C code triggers shape analysis of an AVL tree library.
The code starts from a valid AVL tree, inserts one random integer
into it then proves that the AVL tree is still valid after
the previous insertion (of any possible integer value).
While such code compiles, it is not meant to be executed.
Such code is used by \memcad to certify that
the considered AVL tree insert function implementation is correct
(only sortedness is not verified).

Example test program to analyze; with an abstract
execution flow (all possible execution flows will be considered):
\begin{verbatim}
void main() {
  // create an uninitialized tree
  jsw_avltree_t* t;
  // assume that it is a valid tree
  _memcad("add_inductive("
            "t,avltree,[ | | ])");
  // insert a number into it
  int random; // uninitialized -> random
  int insert_res = jsw_avlinsert(t,&random);
  // prove the tree is still valid
  _memcad("check_inductive("
            "t,avltree,[ | | ])");
}
\end{verbatim}


\subsection{Analyze a program in the concrete}
The following code tests a unique execution path using an AVL tree library.
The integer one is inserted into an empty tree, then it is found back,
then it is removed and cannot be found back anymore.
Such code compiles, can be executed and dynamically tested even
with valgrind.
Such code ensures that we can write simple example programs using the
data structure library we want to analyze (and that we are not experts at).
Keep in mind that the assert statements are proof goals that \memcad
will try to prove.

Example test program to analyze, with a single concrete
execution flow:
\begin{verbatim}
void main() {
  // allocate a tree
  jsw_avltree_t* t = jsw_avlnew();
  // insert one into it
  int one = 1;
  int insert_res = jsw_avlinsert(t, &one);
  assert(insert_res == 1); // success ?
  // find one
  int* res = NULL;
  res = jsw_avlfind(t, &one);
  assert(*res == 1); // found ?
  // remove one
  int erase_res = jsw_avlerase(t, &one);
  assert(erase_res == 1); // removed ?
  res = jsw_avlfind(t, &one);
  assert(res == NULL); // can't find anymore
}
\end{verbatim}

