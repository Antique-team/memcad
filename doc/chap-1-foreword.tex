%--------------------------------------------------------------------
% Foreword: audience, authors, licence,, installation
%--------------------------------------------------------------------
\chapter{Foreword}
\label{c:1:foreword}

\section{Audience}
This document is intended at users of the \memcad static analyzer.
General information on the \memcad project are available on the main
project website:
\begin{center}
  \url{http://www.di.ens.fr/~rival/memcad}
\end{center}
For contributors to \memcad, there exists a mailing list:
\begin{center}
  \url{memcad@ens.fr}
\end{center}
The list homepage can be found at:
\begin{center}
  \url{https://lists.ens.fr/wws/info/memcad}
\end{center}
For more information, or in order to join the mailing list, please contact
Xavier Rival (\texttt{xavier.rival [ at ] ens.fr}).
Joining the mailing list ensures that you are aware of current developments
in the tool, new features and any bugs discovered and/or corrected.

\section{Contributors}
The \memcad analyzer is developed in the ``ANTIQUE'' (ANalyse staTIQUE)
project team, at INRIA Paris.
Over the years, the following people listed in alphabetical order
have contributed to the source code of \memcad:
\begin{itemize}
\item Berenger Francois
\item Li Huisong
\item Liu Jiangchao
\item Rival Xavier
\item Sotin Pascal
\item van Steenhoven Pippijn
\item Toubhans Antoine
\end{itemize}

\section{License}
The source code of the \memcad analyzer is licensed under the Free
Software Foundation GNU General Public License, either version 3 or
(at your option) any later version.
%% TODO: should we not freeze the license version ?

\section{Installation}
This section will guide the user through the installation of the \memcad
analyzer under a Linux environment using the command line, and basic
utilities.
It describes the required dependencies and the compilation from sources.

The installation requires the following dependencies:
\begin{itemize}
\item OPAM, version $>=$ 1.2.0, which can be found (together with
  instructions for its installation and configuration) at:
  \begin{center}
    \url{https://opam.ocaml.org/}
  \end{center}
\item OCaml, version $>=$ 4.02.3
\end{itemize}

\subsection{OPAM configuration}
We advise novice OCaml users to install both OPAM and OCaml
at the same time.
This can be done through the commands:
\begin{verbatim}
$ wget https://raw.github.com/ocaml/\
$ opam/master/shell/opam_installer.sh
$ mkdir -p ~/bin
$ export PATH=$PATH:~/bin
$ sh opam_installer.sh ~/bin 4.02.3
$ opam init
$ eval `opam config env`
\end{verbatim}
Please refer to
\begin{center}
  \url{https://opam.ocaml.org/doc/Install.html}
\end{center}
for details and up to date instructions.

For more experienced users, we still advise to let OPAM install OCaml for them.
It is not recommended to use opam with a system-wide installed ocaml compiler.
If you already have OPAM installed, a compiler managed by OPAM
can be installed in user-space with:
\begin{verbatim}
$ opam init
$ opam switch 4.02.3
$ eval `opam config env`
\end{verbatim}

To be able to proceed further, OPAM should be set up
with at least the official default repository.
The command 'opam repo' should return at least:
\begin{verbatim}
0 [http] default https://opam.ocaml.org
\end{verbatim}

The command below configures the default opam repository manually
(if it is not already):
\begin{verbatim}
$ opam repo add default https://opam.ocaml.org
\end{verbatim}

Before continuing with the installation, opam packages should be upgraded
using the following commands:
\begin{verbatim}
$ opam update
$ opam upgrade
\end{verbatim}

\subsection{Regular user install}
This subsection is for users who just want to install and use
an officially released version of \memcad.

\memcad is now ready to install:
\begin{verbatim}
$ opam install memcad
\end{verbatim}

\memcad is a complex software with several dependencies.
Some of those dependencies require specific system packages
to be installed (clang for example).
For any \memcad dependency failing to install (let's say dependency D
failed); use the following command:
\begin{verbatim}
$ opam depext -i D
\end{verbatim}

Then try again:
\begin{verbatim}
$ opam install memcad
\end{verbatim}

\subsection{Developer install}
This subsection is for users who want to install \memcad from sources.

First, overriding the default memcad package is necessary:
\begin{verbatim}
$ git clone ssh.di.ens.fr:~rival/git/memcad.git
$ cd memcad
$ opam pin add memcad $PWD
\end{verbatim}
This will download and install all required dependencies
then compile \memcad from sources.

In case the automatic dependencies installation and compilation of \memcad
failed, the following list of commands can be tried:
\begin{verbatim}
$ git clone ssh.di.ens.fr:~rival/git/memcad.git
$ cd memcad
$ opam depext clangml
$ opam depext apron
$ opam install \
  dolog obuild apron parmap clangml \
  clangml-transforms bdd ounit qtest setr
$ make
\end{verbatim}

\subsection{Test suite}
Two regression test suites are provided with the analyzer and can be
analyzed automatically:
\begin{verbatim}
$ make prtp
$ make bigrt
\end{verbatim}
Not a single test should end in a failure or timeout state (unless
running with a particularly slow hardware, in which case, the bounds
for timeouts can be increased ---as explained in \cref{3:refman}).
